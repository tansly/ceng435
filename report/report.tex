\documentclass[conference]{IEEEtran}
\IEEEoverridecommandlockouts
% The preceding line is only needed to identify funding in the first footnote. If that is unneeded, please comment it out.
\usepackage{cite}
\usepackage{amsmath,amssymb,amsfonts}
\usepackage{algorithmic}
\usepackage{graphicx}
\usepackage{listings}
\usepackage{textcomp}
\usepackage{xcolor}
\def\BibTeX{{\rm B\kern-.05em{\sc i\kern-.025em b}\kern-.08em
    T\kern-.1667em\lower.7ex\hbox{E}\kern-.125emX}}
\begin{document}

\title{CENG435 Term Project, Part 1\\
}

\author{\IEEEauthorblockN{Doruk Coskun}
\IEEEauthorblockA{
}
\and
\IEEEauthorblockN{Yagmur Oymak}
\IEEEauthorblockA{
}
}

\maketitle

\begin{abstract}
\end{abstract}

\begin{IEEEkeywords}
ceng, networking, delay
\end{IEEEkeywords}

\section{Introduction}

In our CENG435 Term Project Part 1 we have designed a network topology according to the information provided to us in the homework description. We have formed a TCP connection between Source and Broker. We implemented UDP connection between Broker, Router 1, Router 2 and Destination. We tested packet loss and packet delays in our network.

\section{Setting Up the Environment}
To set network emulation delays, \textbf{tc} command is used. \textbf{tc} manipulates
traffic control parameters in a Linux system. Running verbose \textbf{tc} commands
in multiple machines becomes tedious very quickly.
In order to quickly and conveniently set delays in multiple nodes and multiple links,
we've used the shell scripts mentioned in the README file.

Likewise, short shell scripts were used for placing the code files in relevant
nodes.

\section{Network Topology Design Decisions}

In our design, packet sent from Source passes through Broker, Router 1 and reaches Destination. All scripts except Broker is written in Python3. Before the packet is sent a TCP connection formed between Source and Broker. Our sensor readings are 8 bytes long and consists of the index of the packet. We have chosen this packet size so that we can can minimize the transmission delay and observe the netem/tc delays better. When the byte stream reaches Broker, they are sent to Router 1 through UDP connection. Router 1 listens to a predefined port number and directs the packet to Destination. Destination prints out the content of the received packet and sends feedback back to the source through Router 2. Router 2 directs the packet received from Destination to Broker. Finally broker uses the already open TCP connection to send the feedback message to Source. Source calculates the time difference between the sent packet and the received feedback message. By this method we have calculated the round-trip time. We have also implemented a method where we have synchronized the nodes and calculated the one way delay using time stamps that are sent with the packets. But we believe we have gathered more accurate results using round-trip time. You can find detailed synchronization implementation in the following parts of this documentation.

\subsection{Test 1: Packet Loss Test}\label{AA}

Apart from delay tests we have also implemented packet loss test. In this test we have sent 1000 consecutive packets from Source and observed how many of them reached to Broker and then to Destination. Its important to keep in mind that our packet size is 8 bytes. 
\begin{itemize}
\item 1000/1000: All of the packets sent by Source were received by Broker. Since the connection between Source and Destination is TCP, packet loss is not expected.
\item 67/1000: Only 67 of the packets were recieved by the destination. There is a significant packet loss. We exceed the capacity of the links between the other nodes. That's why while transfering the packets through UDP nodes, most of them were lost. While we were sending packets of size 128 bytes, the packet loss rate was even higher, with only 17 packets reaching Destination. We also observed that when we increased the size of the packets to 4 times, the amount of the packets delivered to Destination reduced to 1 over 4.
\end{itemize}

\subsection{Detailed Explanation of Synchronized Nodes}
As previously mentioned, we approximate the end-to-end delay using the round trip time, i.e.\
end-to-end delay is round trip time divided by two. Since our network delays will
most likely be symmetric, this approach deemed to be accurate enough. Still,
we implmented an alternative method of calculating the end-to-end delay using
timestamps in packets. For this approach to produce reliable results, the clocks
of the source and the destination must be synchronized.

In order to synchronize the nodes, \textbf{NTP} is used. NTP sets the clock of the
node using information from a time server. The following command is used to set the
clock of the node:
\begin{lstlisting}
sudo ntpdate -s time.nist.gov
\end{lstlisting}
Then, the \textbf{clockdiff} program can be used to calculate the clock difference
of two nodes. \textbf{clockdiff} uses ICMP TIMESTAMP packets (RFC0792, page 16)
to measure the clock difference of two hosts with a 1ms resolution. An example run:
\begin{lstlisting}
yagmuroy@d:~\$ clockdiff 172.17.1.7.
host=172.17.1.7 rtt=750(187)ms/0ms
delta=-277ms/-277ms Wed Nov 28 11:55:06 2018
\end{lstlisting}
When the clock difference (delta) became small (around 5ms), we ran our experiments.
Not to our surprise, the measured delay value was very close to our other approximation
(round trip time divided by two). For practical purposes\footnote{Even though the nodes had been
synchronised, their clocks started to drift and introduced noise in our measurements
before any meaningful data was collected.}, we deemed it appropriate
to use the round trip time approximation instead of the timestamp method.

\subsection{Detailed Explanation of Broker Implementation}
The broker is implemented as a multi-process server in C. This section describes
the implementation of the broker from a high level viewpoint.

The broker, when started, creates a TCP socket, binds it to a local port, and listens
it. It also creates two UDP sockets, one to send datagrams to Router 1, and one to receive
datagrams from Router 2. Since the first UDP socket will always send datagrams to a
fixed address and port (identifying the receiving socket at Router 1), we also
call connect on the socket to save us from specifying the address and port at every send call.
The second will receive datagrams from Router 2, therefore it must wait for packets
on a fixed port (the port that Router 2 will send to). So it also binds the socket
to a local address and port.

After the sockets are created, connected, bound and listened, the broker starts
its main loop in which it accepts TCP connections with the accept system call.
When the Source connects to the broker, a new worker process is spawned. This worker
process will handle the TCP connection socket. It receives byte streams from the Source,
packetizes the bytes and sends them to Router 1 using the previously
created UDP sockets. It also spawns another worker process (which shares the connection
socket) that receives datagrams (containing feedback messages) from Router 2 and
sends them over the TCP connection socket to the Source.

When the TCP connection from the source is terminated, the recv call will return 0
to the broker indicating connection termination. Knowing the connection is closed,
the worker process kills its child (the other worker process that it spawned),
checks if there is any leftover data coming from Router 2, clears it if there is
and exits. The main process still waits to accept new TCP connections, so the brokes
is a server that always keeps running.

\subsection{\LaTeX-Specific Advice}

Please use ``soft'' (e.g., \verb|\eqref{Eq}|) cross references instead
of ``hard'' references (e.g., \verb|(1)|). That will make it possible
to combine sections, add equations, or change the order of figures or
citations without having to go through the file line by line.

Please don't use the \verb|{eqnarray}| equation environment. Use
\verb|{align}| or \verb|{IEEEeqnarray}| instead. The \verb|{eqnarray}|
environment leaves unsightly spaces around relation symbols.

Please note that the \verb|{subequations}| environment in {\LaTeX}
will increment the main equation counter even when there are no
equation numbers displayed. If you forget that, you might write an
article in which the equation numbers skip from (17) to (20), causing
the copy editors to wonder if you've discovered a new method of
counting.

{\BibTeX} does not work by magic. It doesn't get the bibliographic
data from thin air but from .bib files. If you use {\BibTeX} to produce a
bibliography you must send the .bib files. 

{\LaTeX} can't read your mind. If you assign the same label to a
subsubsection and a table, you might find that Table I has been cross
referenced as Table IV-B3. 

{\LaTeX} does not have precognitive abilities. If you put a
\verb|\label| command before the command that updates the counter it's
supposed to be using, the label will pick up the last counter to be
cross referenced instead. In particular, a \verb|\label| command
should not go before the caption of a figure or a table.

Do not use \verb|\nonumber| inside the \verb|{array}| environment. It
will not stop equation numbers inside \verb|{array}| (there won't be
any anyway) and it might stop a wanted equation number in the
surrounding equation.

\subsection{Some Common Mistakes}\label{SCM}
\begin{itemize}
\item The word ``data'' is plural, not singular.
\item The subscript for the permeability of vacuum $\mu_{0}$, and other common scientific constants, is zero with subscript formatting, not a lowercase letter ``o''.
\item In American English, commas, semicolons, periods, question and exclamation marks are located within quotation marks only when a complete thought or name is cited, such as a title or full quotation. When quotation marks are used, instead of a bold or italic typeface, to highlight a word or phrase, punctuation should appear outside of the quotation marks. A parenthetical phrase or statement at the end of a sentence is punctuated outside of the closing parenthesis (like this). (A parenthetical sentence is punctuated within the parentheses.)
\item A graph within a graph is an ``inset'', not an ``insert''. The word alternatively is preferred to the word ``alternately'' (unless you really mean something that alternates).
\item Do not use the word ``essentially'' to mean ``approximately'' or ``effectively''.
\item In your paper title, if the words ``that uses'' can accurately replace the word ``using'', capitalize the ``u''; if not, keep using lower-cased.
\item Be aware of the different meanings of the homophones ``affect'' and ``effect'', ``complement'' and ``compliment'', ``discreet'' and ``discrete'', ``principal'' and ``principle''.
\item Do not confuse ``imply'' and ``infer''.
\item The prefix ``non'' is not a word; it should be joined to the word it modifies, usually without a hyphen.
\item There is no period after the ``et'' in the Latin abbreviation ``et al.''.
\item The abbreviation ``i.e.'' means ``that is'', and the abbreviation ``e.g.'' means ``for example''.
\end{itemize}
An excellent style manual for science writers is \cite{b7}.

\subsection{Authors and Affiliations}
\textbf{The class file is designed for, but not limited to, six authors.} A 
minimum of one author is required for all conference articles. Author names 
should be listed starting from left to right and then moving down to the 
next line. This is the author sequence that will be used in future citations 
and by indexing services. Names should not be listed in columns nor group by 
affiliation. Please keep your affiliations as succinct as possible (for 
example, do not differentiate among departments of the same organization).

\subsection{Identify the Headings}
Headings, or heads, are organizational devices that guide the reader through 
your paper. There are two types: component heads and text heads.

Component heads identify the different components of your paper and are not 
topically subordinate to each other. Examples include Acknowledgments and 
References and, for these, the correct style to use is ``Heading 5''. Use 
``figure caption'' for your Figure captions, and ``table head'' for your 
table title. Run-in heads, such as ``Abstract'', will require you to apply a 
style (in this case, italic) in addition to the style provided by the drop 
down menu to differentiate the head from the text.

Text heads organize the topics on a relational, hierarchical basis. For 
example, the paper title is the primary text head because all subsequent 
material relates and elaborates on this one topic. If there are two or more 
sub-topics, the next level head (uppercase Roman numerals) should be used 
and, conversely, if there are not at least two sub-topics, then no subheads 
should be introduced.

\subsection{Figures and Tables}
\paragraph{Positioning Figures and Tables} Place figures and tables at the top and 
bottom of columns. Avoid placing them in the middle of columns. Large 
figures and tables may span across both columns. Figure captions should be 
below the figures; table heads should appear above the tables. Insert 
figures and tables after they are cited in the text. Use the abbreviation 
``Fig.~\ref{fig}'', even at the beginning of a sentence.

\begin{table}[htbp]
\caption{Table Type Styles}
\begin{center}
\begin{tabular}{|c|c|c|c|}
\hline
\textbf{Table}&\multicolumn{3}{|c|}{\textbf{Table Column Head}} \\
\cline{2-4} 
\textbf{Head} & \textbf{\textit{Table column subhead}}& \textbf{\textit{Subhead}}& \textbf{\textit{Subhead}} \\
\hline
copy& More table copy$^{\mathrm{a}}$& &  \\
\hline
\multicolumn{4}{l}{$^{\mathrm{a}}$Sample of a Table footnote.}
\end{tabular}
\label{tab1}
\end{center}
\end{table}

\begin{figure}[htbp]
\caption{Example of a figure caption.}
\label{fig}
\end{figure}

Figure Labels: Use 8 point Times New Roman for Figure labels. Use words 
rather than symbols or abbreviations when writing Figure axis labels to 
avoid confusing the reader. As an example, write the quantity 
``Magnetization'', or ``Magnetization, M'', not just ``M''. If including 
units in the label, present them within parentheses. Do not label axes only 
with units. In the example, write ``Magnetization (A/m)'' or ``Magnetization 
\{A[m(1)]\}'', not just ``A/m''. Do not label axes with a ratio of 
quantities and units. For example, write ``Temperature (K)'', not 
``Temperature/K''.

\section*{Acknowledgment}

The preferred spelling of the word ``acknowledgment'' in America is without 
an ``e'' after the ``g''. Avoid the stilted expression ``one of us (R. B. 
G.) thanks $\ldots$''. Instead, try ``R. B. G. thanks$\ldots$''. Put sponsor 
acknowledgments in the unnumbered footnote on the first page.

\section*{References}

Please number citations consecutively within brackets \cite{b1}. The 
sentence punctuation follows the bracket \cite{b2}. Refer simply to the reference 
number, as in \cite{b3}---do not use ``Ref. \cite{b3}'' or ``reference \cite{b3}'' except at 
the beginning of a sentence: ``Reference \cite{b3} was the first $\ldots$''

Number footnotes separately in superscripts. Place the actual footnote at 
the bottom of the column in which it was cited. Do not put footnotes in the 
abstract or reference list. Use letters for table footnotes.

Unless there are six authors or more give all authors' names; do not use 
``et al.''. Papers that have not been published, even if they have been 
submitted for publication, should be cited as ``unpublished'' \cite{b4}. Papers 
that have been accepted for publication should be cited as ``in press'' \cite{b5}. 
Capitalize only the first word in a paper title, except for proper nouns and 
element symbols.

For papers published in translation journals, please give the English 
citation first, followed by the original foreign-language citation \cite{b6}.

\begin{thebibliography}{00}
\bibitem{b1} G. Eason, B. Noble, and I. N. Sneddon, ``On certain integrals of Lipschitz-Hankel type involving products of Bessel functions,'' Phil. Trans. Roy. Soc. London, vol. A247, pp. 529--551, April 1955.
\bibitem{b2} J. Clerk Maxwell, A Treatise on Electricity and Magnetism, 3rd ed., vol. 2. Oxford: Clarendon, 1892, pp.68--73.
\bibitem{b3} I. S. Jacobs and C. P. Bean, ``Fine particles, thin films and exchange anisotropy,'' in Magnetism, vol. III, G. T. Rado and H. Suhl, Eds. New York: Academic, 1963, pp. 271--350.
\bibitem{b4} K. Elissa, ``Title of paper if known,'' unpublished.
\bibitem{b5} R. Nicole, ``Title of paper with only first word capitalized,'' J. Name Stand. Abbrev., in press.
\bibitem{b6} Y. Yorozu, M. Hirano, K. Oka, and Y. Tagawa, ``Electron spectroscopy studies on magneto-optical media and plastic substrate interface,'' IEEE Transl. J. Magn. Japan, vol. 2, pp. 740--741, August 1987 [Digests 9th Annual Conf. Magnetics Japan, p. 301, 1982].
\bibitem{b7} M. Young, The Technical Writer's Handbook. Mill Valley, CA: University Science, 1989.
\end{thebibliography}
\vspace{12pt}
\color{red}
IEEE conference templates contain guidance text for composing and formatting conference papers. Please ensure that all template text is removed from your conference paper prior to submission to the conference. Failure to remove the template text from your paper may result in your paper not being published.

\end{document}
